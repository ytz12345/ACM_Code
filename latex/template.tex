\documentclass{article}
\usepackage[margin=1in]{geometry}
\usepackage[Glenn]{fncychap}
\usepackage{listings}
\usepackage{color}
\usepackage{verbatim}
\usepackage{zh_CN-Adobefonts_external} 
\usepackage{fancyhdr}  % 页眉页脚
\usepackage{minted}    % 代码高亮
\usepackage[colorlinks]{hyperref}  % 目录可跳转

% 定义页眉页脚
\pagestyle{fancy}
\fancyhf{}
\fancyhead[C]{Algorithm Template Library by ytz123}
\lfoot{}
\cfoot{\thepage}
\rfoot{}

\author{ytz123}   
\title{Algorithm Template Library}

\begin{document} 
\maketitle % 封面
\newpage % 换页
\tableofcontents % 目录

\newpage
\section{注意事项} % 一级标题
\subsection{需要想到的方法} % 二级标题
\inputminted[breaklines]{text}{../注意事项/需要想到的方法.txt}

\subsection{需要注意的问题} % 二级标题
\inputminted[breaklines]{text}{../注意事项/需要注意的问题.txt}

\newpage
\section{图论} % 一级标题
\subsection{线段树维护树直径} % 二级标题
\inputminted[breaklines]{c++}{../图论/线段树维护树直径.cpp} % 插入代码文件

\subsection{有向图判断两个点能否到达} % 二级标题
\inputminted[breaklines]{c++}{../图论/有向图判断两个点能否到达.cpp} % 插入代码文件

\subsection{spfa费用流} % 二级标题
\inputminted[breaklines]{c++}{../图论/spfa费用流.cpp} % 插入代码文件

\subsection{dsu} % 二级标题
\inputminted[breaklines]{c++}{../图论/dsu.cpp} % 插入代码文件

\subsection{长链剖分} % 二级标题
\inputminted[breaklines]{c++}{../图论/长链剖分.cpp} % 插入代码文件

\subsection{LCT} % 二级标题
\inputminted[breaklines]{c++}{../图论/LCT.cpp} % 插入代码文件

\subsection{hungary} % 二级标题
\inputminted[breaklines]{c++}{../图论/hungary.cpp} % 插入代码文件

\subsection{dinic最大流} % 二级标题
\inputminted[breaklines]{c++}{../图论/dinic最大流.cpp} % 插入代码文件

\subsection{DAG删去无用边} % 二级标题
\inputminted[breaklines]{c++}{../图论/DAG删去无用边.cpp} % 插入代码文件

\subsection{树分治} % 二级标题
\inputminted[breaklines]{c++}{../图论/树分治.cpp} % 插入代码文件

\subsection{支配树}
\subsubsection{DAG支配树(含倍增LCA}
\inputminted[breaklines]{c++}{../图论/DAG支配树.cpp}

% \twocolumn  % 分页显示
\newpage
\section{数据结构}
\subsection{splay}
\inputminted[breaklines]{c++}{../数据结构/splay.cpp}

\subsection{treap}
\inputminted[breaklines]{c++}{../数据结构/treap.cpp}

\subsection{主席树}
\inputminted[breaklines]{c++}{../数据结构/主席树.cpp}

\subsection{吉老师线段树}
\inputminted[breaklines]{c++}{../数据结构/吉老师线段树.cpp}

\subsection{KDTree}
\subsubsection{3维KDtree}
\inputminted[breaklines]{c++}{../数据结构/3维KDtree.cpp}

\subsubsection{KDtree二维空间区间覆盖单点查询}
\inputminted[breaklines]{c++}{../数据结构/KDtree二维空间区间覆盖单点查询.cpp}

\subsubsection{KDtree二维空间单点修改区间查询}
\inputminted[breaklines]{c++}{../数据结构/KDtree二维空间单点修改区间查询.cpp}

\subsubsection{KDtree找最近点}
\inputminted[breaklines]{c++}{../数据结构/KDtree找最近点.cpp}

% \twocolumn  % 分页显示
\newpage
\section{构造}
\subsection{若干排列使所有数对都出现一次}
\inputminted[breaklines]{c++}{../构造/若干排列使所有数对都出现一次.cpp}

\subsection{rec-free}
\inputminted[breaklines]{c++}{../构造/rec-free.cpp}

\subsection{点边均整数多边形}
\inputminted[breaklines]{c++}{../构造/点边均整数多边形.cpp}

% \twocolumn  % 分页显示
\newpage
\section{计算几何}
\subsection{最小矩形覆盖含凸包和旋转卡壳}
\inputminted[breaklines]{c++}{../计算几何/最小矩形覆盖含凸包和旋转卡壳.cpp}

% \twocolumn  % 分页显示
\newpage
\section{数论}
\subsection{CRT}
\inputminted[breaklines]{c++}{../数论/CRT.cpp}

\subsection{蔡勒公式}
\inputminted[breaklines]{c++}{../数论/蔡勒公式.cpp}

\subsection{素数判定+大整数质因数分解}
\inputminted[breaklines]{c++}{../数论/素数判定+大整数质因数分解.cpp}

% \twocolumn  % 分页显示
\newpage
\section{字符串}
\subsection{KMP}
\inputminted[breaklines]{c++}{../字符串/KMP.cpp}

% \twocolumn  % 分页显示
\newpage
\section{其他}
\subsection{数字哈希}
\inputminted[breaklines]{c++}{../其他/数字哈希.cpp}

\subsection{海岛分金币}
\subsubsection{海岛分金币1}
\inputminted[breaklines]{c++}{../其他/海岛分金币1.cpp}

\subsubsection{海岛分金币2}
\inputminted[breaklines]{c++}{../其他/海岛分金币2.cpp}

\subsection{根号枚举}
\inputminted[breaklines]{c++}{../其他/根号枚举.cpp}

\subsection{读入输出外挂}
\inputminted[breaklines]{c++}{../其他/读入输出外挂.cpp}

\subsection{给定小数化成分数}
\inputminted[breaklines]{python}{../其他/给定小数化成分数.py}

%\newpage
%\section{Others}

\end{document}